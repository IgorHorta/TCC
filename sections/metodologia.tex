\section{METODOLOGIA}
\label{sec:metod}

\subsection{Extração de dados sobre produtos}

Um dos recursos do sistema proposto é proporcionar a comparação de preços de peças de computador entre diversas lojas de \emph{e-commerce}, para se obter a informação de cada peça com seus respectivos preços e vendedores o sistema utilizara de um algoritmo de extração de dados. Tal algoritmo sera desenvolvido em Javascript e será executado utilizando o PhantomJS.
Os dados que deveram ser extraídos dos \emph{websites} voltados para o \emph{e-commerce} são:

\begin{itemize}
\setlength{\itemsep}{-0.3ex}
\item nome do produto.
\item link da imagem do produto.
\item preço a vista do produto.
\item descrição do produto.
\item link para a compra do produto.
\end{itemize}





faz essa 3.1 o quanto antes
hoje se possível
começa explicando o que você precisa extrair de informação
depois explica como você vai fazer isso
explica o quê você precisa e explica também o porquê você precisa delas
aí fala como vai extraí-las, explicando detalhadamente o algoritmo e o processo que pretende utilizar
