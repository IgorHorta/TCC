\section{METODOLOGIA}
\label{sec:metod}

\subsection{Extração de dados sobre produtos}

\citeonline{alonso2015agente} realizou um estudo sobre extração de preços em sites de \emph{e-commerce}. Neste trabalho foi desenvolvido um agente de \emph{Shopping Comparison} que utilizava padrões em documentos HTML para obter produtos e seus respectivos atributos.Porém este estudo possuía o ponto deficiente de não conseguir obter informações de \emph{websites} que utilizam a linguagem Javascript para exibir o preços dos produtos assincronamente,o que é uma prática comum em sites atuais.

O presente trabalho tratará os pontos deficientes do estudo de \citeonline{alonso2015agente}, e aspectos da aplicação em si.

Um dos recursos do sistema proposto é proporcionar a comparação de preços de peças de computador entre diversas lojas de \emph{e-commerce}, para se obter a informação de cada peça com seus respectivos preços e vendedores o sistema utilizara de um algoritmo de extração de dados. Tal algoritmo sera desenvolvido em Javascript e será executado utilizando o PhantomJS.
Os dados que deveram ser extraídos dos \emph{websites} voltados para o \emph{e-commerce} são:

\cite{alonso2015agente}

\begin{itemize}
\setlength{\itemsep}{-0.3ex}
\item nome do produto.
\item link da imagem do produto.
\item preço do produto.
\item descrição do produto.
\item link para a compra do produto.
\end{itemize}


As informações do produto como nome, link da imagem, preço, link para a compra serão necessárias para identificar o produto, exibir sua imagem e manter informações sobre o vendedor.


Para se obter tais informações o scrpit primeiramente ira obter a árvore DOM do \emph{website} e ai então utilizando esta a DOM juntamente com seletores    


faz essa 3.1 o quanto antes
hoje se possível
começa explicando o que você precisa extrair de informação
depois explica como você vai fazer isso
explica o quê você precisa e explica também o porquê você precisa delas
aí fala como vai extraí-las, explicando detalhadamente o algoritmo e o processo que pretende utilizar
