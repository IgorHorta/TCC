\section{RESULTADOS E DISCUSS�O}
\label{sec:results}


\subsection{Cronograma}
1 - Estudar sobre algoritmos de detec��o de padr�es em p�ginas HTML (1 m�s)

2 - Desenvolver um algoritmo eficiente para reconhecimento de padr�es em HTML, que trate 
altera��es na p�gina com Javascript (2 meses)

3 - Desenhar e planejar as telas da aplica��o (15 dias)

3 - Modelagem do banco de dados para a aplica��o (15 dias)

4 - Implementa��o do \emph{back-end} da aplica��o (1 m�s)

5 - Implementa��o do \emph{front-end} da aplica��o (1 m�s)

\subsection{Resultados Esperados}

Ao final deste trabalho espera-se um algoritmo que seja capaz de extrair pre�os e demais informa��es de pelo menos 5 sites de \emph{e-commerce} diferentes, sendo que pelo menos 2 deles utilizem a linguagem Javascript para atualizar as informa��es dos produtos assincronamente.


Tamb�m espera-se um prot�tipo de uma aplica��o \emph{web} que utilize os dados obtidos pelo algoritmo de extra��o de dados. Tal aplica��o deve ter como objetivo comparar os pre�os dos produtos obtidos pelo algoritmo.